\documentclass[a4paper, twoside, 12pt]{artikel3} %jos artikel3 ei toimi, käytä article

\usepackage[T1]{fontenc}
\usepackage[utf8]{inputenc}
\usepackage[english]{babel} %katso että kääntäjä osaa kans tavuttaa, jos useita kieliä, default-kieli viimeiseksi
\usepackage{amsmath}
\usepackage{amsfonts}
\usepackage{icomma} %käytä pilkkua kaavoissa ja taulukoissa
\usepackage{enumerate}
\usepackage[hmargin={35mm, 35mm}, vmargin={25mm, 25mm}]{geometry} %en ole varma tarviiko
\usepackage{hyperref} %pdf-tiedostoon viittaukset hyperlinkkeinä, linkkien ympärille tulevat jutut eivät tule tulosteeseen.
\usepackage{verbatim} %pitkiin kommentteihin...
\usepackage{url} %jotta LaTeX ymmärtää url-osoitteet (\url-komento)
\usepackage{graphicx}


\newcommand{\degree}{^{\circ}} %Astemerkki matematiikkatilassa
\renewcommand{\vec}{\mathbf} %Vektorin ylänuolen sijasta vektorit lihavoituna
\newcommand{\doo}{\partial}
\newcommand{\dee}{\mathrm{d}}
\newcommand{\email}{\href{mailto:olli.vaisanen@helsinki.fi}{olli.vaisanen@helsinki.fi}} %Oma email-osoite komentona :)
\newcommand{\emailto}[1]{\href{mailto:#1}{#1}}
\renewcommand{\dateseparator}{.} %Vaihtaa \today-komennon päivämääräerotinta (datetime-paketin default on '/' )
\newcommand{\kaari}[1]{\left( #1 \right)}
\newcommand{\aalto}[1]{\left{ #1  \right}}
\newcommand{\haka}[1]{\left[ #1 \right]}


\begin{document}

\pagestyle{empty} 
\pagenumbering{none}
[NAME OF THE COURSE] \hfill Name: Olli Väisänen\\%Kurssin nimi
[Spring/Fall 20XX] \hfill e-mail: \href{mailto:olli.vaisanen@helsinki.fi}{olli.vaisanen@helsinki.fi} \\
Exercise  \hfill  Student id: 013326836 \\%HARJOITUKSEN NUMERO!

\begin{enumerate}[Prob. 1.] \itemsep12pt
	\item %Tehtävä 1, seuraavat tehtävät jälleen itemeinä.
		%\begin{enumerate}[(a)] \itemsep8pt
			%Tähän tehtävän osat a), b), c), etc. itemeinä.
				%\begin{enumerate}[(i)] \itemsep6pt
					%Jos tehtävällä vielä aliosia, niin jatketaan enumerateja...
				%\end{enumerate}
		%\end{enumerate}
\end{enumerate}

\begin{comment}
Kappaleen vaihto tyhjällä rivillä!:

Jos jossain kohdassa ei saa vaihtaa riviä, käytä tildeä välilyönnin sijaan:~
Esim: Tulokseksi saatiin~23~metriä~\cite{MAOL}~.

Dokumentin sisäiset viitteet:
\label{avain}-komennolla voidaan kuvien ja kaavojen ym. lisäksi labeloida melkein kaikkea (esim. kokonaisia kappaleita, osioita, ym.).
Labelin kohteeseen voidaan viitata \ref{avain}-komennolla. \eqref{eq:avain} kaavoille (laittaa sulut viittauksen ympärille)

Leijuvat kuvat:
\begin{figure}[!h]
	\begin{center}
		\includegraphics[width=0.5\textwidth]{Scarlett.png}
	\end{center}
	\caption{Oh Scarlett! $\heartsuit$}
	\label{fig:Scarlett}
\end{figure}


Matematiikka:

Tekstin osana:
Komennolla $Matematiikkaa...$ saadaan matematiikkaa ladottuna tekstin sekaan

Erillisinä kaavoina:
\begin{equation} 
	\label{eq:avain}
	Matematiikkaa...
\end{equation}
Kaava sijoitetaan erikseen ja numeroidaan. Numeroimaton versio saadaan equation*:llä. 
Kaavaan viitataan: \eqref{eq:avain}, esim: ...keskiarvo saadaan kaavalla~\eqref{eq:karvo}~.
Kaavaan saadaan tarvittaessa tekstin tavoin muotoiltuja merkkejä (esim. pilkku ennen missä-sanaa)
komennolla \text{teksti}, esim: \text{,}
Monen rivin kaava saadaan käyttämällä multiline-ympäristöä equation-ympäristön sijaan. Monen rivin kaavassa voidaan
vaihtaa riviä \\-komennolla

\begin{align} 
	3x+2y&=3\\
	-2x&=4y+7\notag %notagilla voidaan jättää yksi yhtälö numeroimatta
\end{align} %align-ympäristöllä saa kohdistettuja yhtälöryhmiä, &-merkillä merkitään kohdistuskohta, \\ vaihtaa riviä numeroi yhtälöt, align*-ympäristö jättää kaikista numeroinnit

f(x) =
\begin{cases} 
0 & \text{ jos } x<0 \\
x & \text{ jos } 0\le x \le 0 \\
1 & \text{ jos } 1 > x
\end{cases} 
%cases-ympäristöllä saadaan paloittain määritelty funktio, sitä pitää käyttää jonkin toisen matikkaympäristön sisällä. Huomaa kiinnitysmerkit & sekä välit \text-komennon sisällä

\begin{pmatrix}
a & b & c \\
d & e & f
\end{pmatrix} 
%pmatrix luo matriisin (pyöreät sulut), bmatrix tekee hakasulkumatriiseja, etc

\left(\begin{array}{c|cc}
a & b & c \\
\hline d & e & f
\end{array}\right) %array on vanha tapa luoda matriiseja, sillä saadaan mm. lohkomatriiseja edellä esitetyllä tavalla

Matematiikan komentoja:
\: -komennolla saadaan matikkatilassa pienen välin merkkien väliin
\, \quad ja \qquad antavat eripituisia välejä matikkatilassa, \! vähentää merkkien välistä tilaa, voidaan käyttää esim moni-integraalien kanssa tiivistämään merkintöjä, esim: \int\!\!\!\int
Kreikkalaiset aakkoset: \ja aakkosen englanninkielinen nimi. Isolla alkukirjaimella iso, pienellä pieni aakkonen
Esim: \alpha pikkualfalle, \Delta isolle deltalle
_ alaindekseille ja ^ yläindekseille, voi yhdistellä vapaasti, eivät mene sekaisin. Pidemmät jutut aaltosulkuihin.
Esim: x^2_{max}
\infty äärettömän merkille
Murtoluvut ja jakolaskut: \frac{}{}, aaltosulkuihin viivan ylä- ja alapuolille tulevat jutut. Osaa venyä fiksusti.
Summamerkki: \sum_{alaindeksi}^{yläindeksi}
Neliöjuuri: \sqrt{}, venyy järkevästi
Järkevästi venyvät sulut: \left(, \right) sekä \left[, \right] ja \left{, \right}  
Keskiarvoa merkitsevä viiva kirjaimen päälle: \bar{}
Plusmiinus: \pm
Noin yhtäsuuri kuin: \approx
Pienempi tai yhtäsuuri kuin(?): \le
Kertomerkki näkyviin: \times
Doo-merkki: \partial
Sopivasti venyvät merkit: \right ja \left (esim. \right), \left[, \right/, etc) pisteellä (.) voi rajata merkkiä 

Kolme pistettä: \ldots


Omien komentojen luonti:
\newcommand{\RR}{\mathbb{R}} %voidaan luoda vain "lyhennysmerkkejä" TAI
\newcommand{\MB}[2]{\mathbb{#1}-\mathbb{#2}} %näin voidaan luoda "funktion tavoin" toimivia komentoja. Hakasuluissa on argumenttien määrä ja #-merkillä merkataan argumenttinä saadut muuttujat
Käyttö: Ensimmäiselle normaali \RR, jälkimmäiselle esim. \MB{R}{Y} (tässä tapauksessa matikkatilassa)

\selectlanguage{english} %Vaihda kieltä kesken dokumentin.

\url{http://en.wikipedia.org/wiki/Scarlett_Johansson} %lisää linkin nettiosoitteeseen, komento tulee hyperref-paketissa

\footnote{Sivuhuomautus} %sivun sisäinen viite

Numeroimattomat listat:
\begin{itemize} %itemize-ympäristössä listan kohdan eteen tulevaa merkkiä voi muuttaa item-kohtaisesti hakasulkeisiin tulevalla merkillä (ks. seuraavassa)
\item[-] Esko
\item[a] on 
\item kiki 
\end{itemize}


\end{comment}

\end{document}
