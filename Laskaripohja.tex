\documentclass[a4paper, twoside, addpoints, 12pt]{exam} %jos artikel3 ei toimi, käytä article

\usepackage[T1]{fontenc}
\usepackage[utf8]{inputenc}
\usepackage[finnish]{babel} %katso että kääntäjä osaa kans tavuttaa, jos useita kieliä, default-kieli viimeiseksi
\usepackage{lmodern} %hiukan uudempi, parempi versio LaTeXin perusfontista
\usepackage{textcomp} %tarjoaa ylimääräisiä tekstisymboleja, kuten \texteuro, \textcelsius
\usepackage{mathtools, amssymb} %uudempi matematiikkapaketti mathtools (sisältää amsmathin), amssymb lisää matikkasymboleita
\usepackage{amsfonts}
\usepackage{icomma} %käytä pilkkua kaavoissa ja taulukoissa (Suom. käytäntö)
\usepackage{cancel} %Termien yliviivaus matematiikkatilassa (komennolla \cancel{})
\usepackage[shortlabels]{enumitem} %enumitem korvaa vanhentuneen enumeraten, shortlabels mahdollistaa labelien muokkauksen
\usepackage[hmargin={35mm, 35mm}, vmargin={25mm, 25mm}]{geometry} %marginaalien säätöä...
\usepackage[pdftex]{graphicx} %kuvaajapaketti
\usepackage[pdftex]{color} %mahdollistaa värillisen tekstin, 
\usepackage[pdftex]{hyperref} %pdf-tiedostoon viittaukset hyperlinkkeinä, linkkien ympärille tulevat jutut eivät tule tulosteeseen. Lisää myös kätevät linkkipaketit (ks. käyttö lopusta)
\usepackage{verbatim} %korjaa verbatim-ympäristön bugeja ja lisää comment-ympäristön pitkille kommenteille
\usepackage[footnotesize, bf]{caption} %kuvien ja taulukoiden kuvatekstin muokkaus (esim. pienempi fontti, lihavointi)
\usepackage{datetime} %muokkaa \today komennon formaattia
\usepackage{siunitx} %numeroiden ja erityisesti yksiköiden muotoilu konsistentiksi, katso ohjeita lopusta, toimii sekä matematiikkatilassa että tekstissä
\sisetup{	%siunitx-paketin asetukset vaativat suomenkielessä vähän säätöä
	per-mode = symbol-or-fraction, %käyttää yksiköiden kuten m/s välissä symbolia / line mathissa ja \frac:ia matematiikkatilassa
	output-decimal-marker = {,},	%suomalainen desimaalieroitin (pilkku)
	list-final-separator = { ja }, %listojen sanallinen eroitin suomeksi
	list-pair-separator = { ja },
	range-phrase = {--},	%defaultina lukuvälit ilmoitetaan "2 to 10"
	separate-uncertainty = true}	%tuloksen epävarmuus erillisenä, esim. (12.3 ± 0.4) kg
\DeclareSIUnit\Aa{\text{Å}}	%Å ei ole enää SI-yksikkö, joten siunitx-paketti kikkaa komennon \angstrom käytöstä. Määritellään oma

\newcommand{\degree}{^{\circ}} %Astemerkki matematiikkatilassa
\renewcommand{\vec}{\mathbf} %Vektorin ylänuolen sijasta vektorit lihavoituna
\newcommand{\doo}{\partial}
\newcommand{\dee}{\mathrm{d}}
\newcommand{\email}{\href{mailto:olli.vaisanen@helsinki.fi}{olli.vaisanen@helsinki.fi}} %Oma email-osoite komentona :)
\newcommand{\emailto}[1]{\href{mailto:#1}{#1}}
\renewcommand{\dateseparator}{.} %Vaihtaa \today-komennon päivämääräerotinta (datetime-paketin default on '/' )
\newcommand{\kaari}[1]{\left( #1 \right)}
\newcommand{\aalto}[1]{\left{ #1  \right}}
\newcommand{\haka}[1]{\left[ #1 \right]}

\newcommand{\LAMMPS}{\textsf{LAMMPS} }

\pagestyle{head}	%HUOM! jos käyttää ilma pagenumberingia (none), niin exam documentclass menee rikki (haluaa käyttää numerointia monessa komennossa...), empty toimii
\firstpageheader{[KURSSIN NIMI]\\	%headerin vasen laita
					Kevät/Syksy 20XX\\
					Harjoitus X} %HARJOITUKSEN NUMERO!
					{}								%headerin keskiosa
					{Nimi: Olli Väisänen\\			%headerin oikea laita
					e-mail: \email \\
					Opiskelijanro: XXXXXX}
\firstpageheadrule

\begin{document}
\ddmmyyyydate %muuttaa (yhdessä dateseparatorin vaihtamisen kanssa) today-komennon muotoilun suomalaiseksi: DD.MM.YYYY

%%%%%%Header jos laskaripohja on documentclass article tai artikel3
%\pagestyle{empty} 
%\pagenumbering{none}
%[Kurssi] \hfill Nimi: Olli Väisänen\\%KURSSIN NIMI!
%Syksy/Kevät 201X \hfill e-mail: \email \\
%Harjoitus X \hfill Opiskelijanro: XXXXXXXXX \\%HARJOITUKSEN NUMERO!

\begin{questions}	%exam dokumenttiluokan ympäristö, numeroi tehtävät järkevämmin (tosin enumitems ois tehny tän kyl kans..)

	\question 
	
%	\question
%	\begin{parts}
%		\part
%		\begin{subparts}
%			\subpart
%		\end{subparts}
%	\end{parts}

\end{questions}

%%%%%%Tehtävien numerointi jos documenttiluokka on article tai artikel3
%\begin{enumerate}[Teht 1.] \itemsep12pt
%	\item %Tehtävä 1, seuraavat tehtävät jälleen itemeinä.
%		%\begin{enumerate} \itemsep8pt
%			%\item%Tähän tehtävän osat a), b), c), etc. itemeinä.
%				%\begin{enumerate} \itemsep6pt
%					%\item %Jos tehtävällä vielä aliosia, niin jatketaan enumerateja...
%				%\end{enumerate}
%		%\end{enumerate}
%\end{enumerate}

\begin{comment}
Kappaleen vaihto tyhjällä rivillä!:

Jos jossain kohdassa ei saa vaihtaa riviä, käytä tildeä välilyönnin sijaan:~
Esim: Tulokseksi saatiin~23~metriä~\cite{MAOL}~.

Muista: Lainausmerkit näin ``lainaus''

Dokumentin sisäiset viitteet:
\label{avain}-komennolla voidaan kuvien ja kaavojen ym. lisäksi labeloida melkein kaikkea (esim. kokonaisia kappaleita, osioita, ym.).
Labelin kohteeseen voidaan viitata \ref{avain}-komennolla. \eqref{eq:avain} kaavoille (laittaa sulut viittauksen ympärille)

Kuvien liittäminen:

Leijuvainen kuva luodaan figure-ympäristöllä:
\begin{figure}[sijoittelu]
	\begin{center}
		\includegraphics[määreitä?]{kuvatiedoston nimi}
	\end{center}
	\caption{kuvateksti}
	\label{fig:avain}
\end{figure}
Ympäristö luo leijuvaisen kuvan jonka sijoittelusta kääntäjä päättää. Kuva numeroidaan ja 
siihen voi viitata: \ref{fig:avain}. Kuvan sijoitteluun voi (yrittää) vaikuttaa sijoittelumääreillä:
	h tähän
	t jonkun läheisen sivun ylälaitaan
	b jonkun läheisen sivun alalaitaan
	p erilliselle leijuvaissivulle
	! lisäämällä huutomerkin LaTeX ohittaa jotain sijoittelun rajoituksia
Määreitä voi yhdistellä, esim: \begin{figure}[htp]. Järjestys vaikuttaa siihen missä järjestyksessä LaTeX kokeilee eri vaihtoehtoja
\includegraphics-komennolle annettavilla määreillä voi mm. skaalata ja kääntää kuvaa:
"angle="-määre kääntää kuvaajaa annetun astemäärän pos-neg kiertosuunta voimassa, 
"width="-määre skaalaa kuvaajan annetun levyiseksi. Ymmärtää mm. \textwidth-komennon, eli skaalaa kuvan tekstin
levyiseksi. Voi myäs käyttää esim: 0.75\textwidth


Matematiikka:

Tekstin osana:
Komennolla $Matematiikkaa...$ saadaan matematiikkaa ladottuna tekstin sekaan

Erillisinä kaavoina:
\begin{equation} 
	\label{eq:avain}
	Matematiikkaa...
\end{equation}
Kaava sijoitetaan erikseen ja numeroidaan. Numeroimaton versio saadaan equation*:llä. 
Kaavaan viitataan: \eqref{eq:avain}, esim: ...keskiarvo saadaan kaavalla~\eqref{eq:karvo}~.
Kaavaan saadaan tarvittaessa tekstin tavoin muotoiltuja merkkejä (esim. pilkku ennen missä-sanaa)
komennolla \text{teksti}, esim: \text{,}
Monen rivin kaava saadaan käyttämällä multiline-ympäristöä equation-ympäristön sijaan. Monen rivin kaavassa voidaan
vaihtaa riviä \\-komennolla

\begin{align} 
	3x+2y&=3\\
	-2x&=4y+7\notag %notagilla voidaan jättää yksi yhtälö numeroimatta
\end{align} %align-ympäristöllä saa kohdistettuja yhtälöryhmiä, &-merkillä merkitään kohdistuskohta, \\ vaihtaa riviä numeroi yhtälöt, align*-ympäristö jättää kaikista numeroinnit

f(x) =
\begin{cases} 
0 & \text{ jos } x<0 \\
x & \text{ jos } 0\le x \le 0 \\
1 & \text{ jos } 1 > x
\end{cases} 
%cases-ympäristöllä saadaan paloittain määritelty funktio, sitä pitää käyttää jonkin toisen matikkaympäristön sisällä. Huomaa kiinnitysmerkit & sekä välit \text-komennon sisällä

\begin{pmatrix}
a & b & c \\
d & e & f
\end{pmatrix} 
%pmatrix luo matriisin (pyöreät sulut), bmatrix tekee hakasulkumatriiseja, etc

\left(\begin{array}{c|cc}
a & b & c \\
\hline d & e & f
\end{array}\right) %array on vanha tapa luoda matriiseja, sillä saadaan mm. lohkomatriiseja edellä esitetyllä tavalla

Matematiikan komentoja:
\: -komennolla saadaan matikkatilassa pienen välin merkkien väliin
\, \quad ja \qquad antavat eripituisia välejä matikkatilassa, \! vähentää merkkien välistä tilaa, voidaan käyttää esim moni-integraalien kanssa tiivistämään merkintöjä, esim: \int\!\!\!\int
Kreikkalaiset aakkoset: \ja aakkosen englanninkielinen nimi. Isolla alkukirjaimella iso, pienellä pieni aakkonen
Esim: \alpha pikkualfalle, \Delta isolle deltalle
_ alaindekseille ja ^ yläindekseille, voi yhdistellä vapaasti, eivät mene sekaisin. Pidemmät jutut aaltosulkuihin.
Esim: x^2_{max}
\infty äärettömän merkille
Murtoluvut ja jakolaskut: \frac{}{}, aaltosulkuihin viivan ylä- ja alapuolille tulevat jutut. Osaa venyä fiksusti.
Summamerkki: \sum_{alaindeksi}^{yläindeksi}
Neliöjuuri: \sqrt{}, venyy järkevästi
Järkevästi venyvät sulut: \left(, \right) sekä \left[, \right] ja \left{, \right}  
Keskiarvoa merkitsevä viiva kirjaimen päälle: \bar{}
Plusmiinus: \pm
Noin yhtäsuuri kuin: \approx
Pienempi tai yhtäsuuri kuin(?): \le
Kertomerkki näkyviin: \times
Doo-merkki: \partial
Sopivasti venyvät merkit: \right ja \left (esim. \right), \left[, \right/, etc) pisteellä (.) voi rajata merkkiä 

Kolme pistettä: \ldots

SI-yksiköiden muotoilu (siunitx-paketti)
Numerot: \num{3.145}, \numlist{10;30;50;70}, \numrange{10}{30}
Yksiköt: \unit{kg.m/s^2} tai \unit{\kilo\gram\metre\per\square\second}
Molemmat: \qty{1.23 +- 0.5}{J.mol^{-1}.K^{-1}}  HUOM! myös epävarmuuden ilmoittaminen
Myös monimutkaisempia yhdistelmiä löytyy, kuten \qtylist, \qtyproduct ja \qtyrange


Omien komentojen luonti:
\newcommand{\RR}{\mathbb{R}} %voidaan luoda vain "lyhennysmerkkejä" TAI
\newcommand{\MB}[2]{\mathbb{#1}-\mathbb{#2}} %näin voidaan luoda "funktion tavoin" toimivia komentoja. Hakasuluissa on argumenttien määrä ja #-merkillä merkataan argumenttinä saadut muuttujat
Käyttö: Ensimmäiselle normaali \RR, jälkimmäiselle esim. \MB{R}{Y} (tässä tapauksessa matikkatilassa)

\selectlanguage{english} %Vaihda kieltä kesken dokumentin.

\url{http://en.wikipedia.org/wiki/Scarlett_Johansson} %lisää linkin nettiosoitteeseen, komento tulee hyperref-paketissa

\footnote{Sivuhuomautus} %sivun sisäinen viite

Numeroimattomat listat:
\begin{itemize} %itemize-ympäristössä listan kohdan eteen tulevaa merkkiä voi muuttaa item-kohtaisesti hakasulkeisiin tulevalla merkillä (ks. seuraavassa)
\item[-] Esko
\item[a] on 
\item kiki 
\end{itemize}


\end{comment}

\end{document}
