%Tämä on labraselkkaripohja versio 1.20 (päivitetty 01.07.2021)
%Kopiointi ja jakaminen ilman lupaa kielletty
%Kirjoittanut Olli Väisänen (copyright) olli.vaisanen@helsinki.fi
%Ohjeita ja vinkkejä löytyy kommenteista ja lopusta!
\documentclass[a4paper, twoside, 12pt]{artikel3} %jos artikel3 ei toimi, käytä article
												 %pidempiin esim. report, jolloin uloin osataso on \chapter{ }

\usepackage[T1]{fontenc}
\usepackage[utf8]{inputenc}
\usepackage[finnish]{babel} %katso että kääntäjä osaa kans tavuttaa, jos useita kieliä, default-kieli viimeiseksi
\usepackage{lmodern} %hiukan uudempi, parempi versio LaTeXin perusfontista
\usepackage{textcomp} %tarjoaa ylimääräisiä tekstisymboleja, kuten \texteuro, \textcelsius
\usepackage{mathtools} %Korjaa amsmath-pakettia (American mathematical societyn parannus perusmatikkapakettiin) Lataa amsmath-paketin eli sitä ei tarvitse erikseen ottaa käyttöön!
\usepackage{amssymb} %Lisää matikkasymboleita...
\usepackage{icomma} %käytä pilkkua kaavoissa ja taulukoissa (Suom. käytäntö)
\usepackage{cancel} %Termien yliviivaus matematiikkatilassa (komennolla \cancel{})
\usepackage{enumerate} %Numeroidut listat, customoitava!
%\usepackage[sharp]{easylist} %Jos pitää tehdä paljon sisäkkäisiä listoja (ks. käyttö lopusta)
\usepackage[hmargin={35mm, 35mm}, vmargin={25mm, 25mm}]{geometry} %marginaalien säätöä...
\usepackage[pdftex]{graphicx} %kuvaajapaketti
\usepackage[pdftex]{color} %mahdollistaa värillisen tekstin, 
\usepackage[pdftex]{hyperref} %pdf-tiedostoon viittaukset hyperlinkkeinä, linkkien ympärille tulevat jutut eivät tule tulosteeseen. Lisää myös kätevät linkkipaketit (ks. käyttö lopusta)
\usepackage{verbatim} %korjaa verbatim-ympäristön bugeja ja lisää comment-ympäristön pitkille kommenteille
\usepackage[numbers]{natbib} %joustavampaan viittailuun, ks lopusta \citet ja \citep 
\usepackage[footnotesize, bf]{caption} %kuvien ja taulukoiden kuvatekstin muokkaus (esim. pienempi fontti, lihavointi)
\usepackage{datetime} %muokkaa \today komennon formaattia
\usepackage[final]{pdfpages} %lisää sivuja toisesta pdf-dokumentista (esim. mittauspöytäkirjan skannaus)
\usepackage{coffee4} %lisää kahvitahrat selkkariin automaattisesti! ks. käyttö lopusta!

\newcommand{\degree}{^{\circ}} %Astemerkki matematiikkatilassa
%\renewcommand{\vec}{\mathbf} %Vektorin ylänuolen sijasta vektorit lihavoituna
\newcommand{\doo}{\partial}
\newcommand{\dee}{\mathrm{d}}
\newcommand{\email}{\href{mailto:olli.vaisanen@helsinki.fi}{olli.vaisanen@helsinki.fi}} %Oma email-osoite komentona :)
\newcommand{\emailto}[1]{\href{mailto:#1}{#1}}
\newcommand{\tulos}[4]{$#1 = (#2 \pm #3)\mathrm{#4}$}  %mittaustulos muodossa $suure = (x +- y) yks)$
\renewcommand{\dateseparator}{.} %Vaihtaa \today-komennon päivämääräerotinta (datetime-paketin default on '/' )
\newcommand{\kaari}[1]{\left( #1 \right)}
\newcommand{\aalto}[1]{\left{ #1  \right}}
\newcommand{\haka}[1]{\left[ #1 \right]}

\begin{document}

\dmyyyydate %Vaihtaa \today-komennon formaatin muotoa (datetime-paketti)

%%%% Vaihtoehtoinen tapa otsikkosivun tekoon
%\title{OTSIKKO}
%\author{Olli Väisänen \\ \email } % + muut tiedot?
%\date{\today}
%
%\maketitle
%\cleardoublepage

\thispagestyle{empty} 
Olli Väisänen\hfill Mittaus suoritettu:  \\%päiväys!
op.nro XXXXXXXXXX \hfill Assistentti:  \\%assistentti
\email \hfill Palautettu: \today \\%palautuspäivä!
\vfill
\begin{center}
\textbf{\LARGE X. Työn nimi}\\ %työn numero ja nimi! (\LARGE:n perään) muodossa: 1 Veden ominaislämpökapasiteetti
AINEOPINTOJEN LABORATORIOTYÖT I %korjaa II tarvittaessa!
\end{center}
\vfill
%\cofeAm{1}{1}{0}{0}{0}

\clearpage

\pagestyle{plain}

\section*{Tiivistelmä} 
	%_Lyhyt_ kuvaus työstä. Ilmiö, tehdyt mittaukset, päätulokset ja johtopäätökset.
	%Saatua tulosta voi verrata kirjallisuusarvoon.
	
\section{Johdanto}
	%Johdannossa esitellään työn motivaatio. Miksi tätä halutaan mitata? Mitä se opettaa? 
	%Onko esim. historiallisia ongelmia ilmiön tutkimisessa?
	
\section{Koejärjestely}
	%Esitellään mittalaitteisto (laitteiston läpikäynnin tarkkuus riippuu mittauksesta) ja suoritetut mittaukset.
	%Kuvat hyödyllisiä!
	
\section{Teoria}
	%Esitellään työn taustalla oleva teoria. Tarvittavat kaavat (muista lähteet!) esitellään. 
	%Kaikkea ei tarvitse johtaa kuitenkaan alusta. Muista nimetä kaavoissa olevat tunnukset!
	
\section{Tulokset}
	%Tämä ja Johtopäätökset osio kaikkein tärkeimmät!!!
	%Esitellään tulokset tiiviisti ja selkeästi. Mieluiten taulukoina ja graafisina esityksinä.
	%Jokaisesta laskusta yksi kaavaan sijoitus! 
	%Muista otsikoida taulukot ja kuvat niin selkeästi että selittyvät itsenäisesti!!!
	%Tuloksia ilmoitettaessa tulee ilmoittaa myös niiden tarkkuus ja epätarkkuuden vaikutus poikkeamaan 
	%kirjallisuusarvosta.

\section{Johtopäätökset}
	%Selostuksen ehkä tärkein osio!!!
	%Tarkastellaan kuinka hyvin saatu tulos vastaa mielikuvaa tutkitusta ilmiöstä, onko tulos järkevä.
	%Verrataan tulosta teoreettiseen ennusteeseen tai kirjallisuusarvoon. Pohditaan mahdollisia syitä
	%eroavaisuuksiin. Kerrotaan _yksityiskohtaisesti_ mittauksen virhelähteistä (systemaattisista) ja 
	%arvioidaan tuloksen tarkkuutta.

\clearpage


\bibliographystyle{bibtyyli} %Itse luomani BibTeX-tyylitiedosto bibtyyli on suomenkielinen, numeroviitteinen, etunimet lyhentävä ja viittausjärjestykseen perustuva
\bibliography{library} %BibTeX-tiedosto library, josta löytyvät viitteet
%HUOM! .bst- ja .bib-tiedostojen, joita tässä käytetään, täytyy löytyä samasta polusta jossa kääntäminen tehdään, 
%polku on annettava tai on luotava paikallisten TeX-tiedostojen kansio, jonka sijainti annetaan TeXille.
%Ohjeet tällaisen kansion luomiseen oman TeX-distribuutiosi dokumentaatiossa!


%%%%%%%%Tässä alla on lueteltu tapa luoda lähdeluettelo käsin. Sitä voi käyttää jos ei
%%%%%%%%halua/osaa käyttää BibTeXiä.
%\begin{thebibliography}{99} %Lähdeluettelo
%Lähteet luetellaan \bibitem{avain}-komennolla:
%Esim: \bibitem{MAOL} R. Seppänen, M. Kervinen, I. Parkkila, L. Karkela ja P. Meriläinen,
%	 \emph{MAOL-taulukot, matematiikka, fysiikka, kemia}, (Matemaattisten Aineiden Opettajien Liitto MAOL ry,
%	 Otava, 2007)
%	 \bibitem{UF} M. Mansfield, C. O'Sullivan, \emph{Understanding Physics}, (Wiley, Chichester, 1998), s. 411-416
%	 \bibitem{moniste} K. Mizohata, V. Palonen, M. Peura, E. Rauhala, B. Ståhlberg ja K. Österberg, 
%	 \emph{Aineopintojen laboratoriotyöt}, työmoniste, (Helsingin Yliopisto, Fysikkalisten tieteiden laitos, 2005)
%	 \bibitem{WA} E. Weisstein, \emph{Arithmetic Mean}, \emph{MathWorld -- A Wolfram Web Resource}, 
%	 http://mathworld.wolfram.com/ArithmeticMean.html, (luettu 16.11.2009) 
%Viimeinen on esimerkki web-lähteestä.

%\end{thebibliography}

\section*{Liitteet} 
	%Luettelo liitteistä
\begin{enumerate}[L{ii}te A:] %Hakasuluissa muotoilu, Latex osaa luoda luettelon A:sta eteenpäin. Enumerate-paketilla voi vaikuttaa todella joustavasti listan numerointiin.
	%Liitteet luetellaan \item-komennolla.
	%Esim: \item Mittauspöytäkirja
	\item Mittauspöytäkirja
	
\end{enumerate}

\clearpage

%\includepdf[pages=-]{Mittauspoytakirja.pdf} 


\begin{comment}
Kappaleen vaihto tyhjällä rivillä!:

Jos jossain kohdassa ei saa vaihtaa riviä, käytä tildeä välilyönnin sijaan:~
Esim: Tulokseksi saatiin~23~metriä~\cite{MAOL}~.

Typografiaa: eripituiset viivat
tavuviiva - (vain tavuttaessa)
väliviivä -- (esim. "vuosina 1999 -- 2004)
ajatusviiva --- (???)
Kolme pistettä: \ldots
Lainausmerkit: alkaa: ``, loppuu: '' (samat yksinkertaisilla)

Kirjallisuusviittaus: \cite{avain}
Viittaa thebibliographyn viitteeseen: \bibitem{avain}
Kannattaa käyttää tildeä kuten yllä!

\citet[]{} lisää tekijöiden nimet (!) ja viittauksen tekstiin (optiona tekstiä viitaukseen, esim. luku tai sivunrot)
\citep[][]{} voi lisätä tekstiä normaalin numeroviittauksen ympärille, ensin taakse ja sitten eteen (esim. \citep[s. 120-143]{avain}-> [3, s. 120-143], \citep[ks.][]{avain}-> [ks. 3])

Dokumentin sisäiset viitteet:
\label{avain}-komennolla voidaan kuvien ja kaavojen ym. lisäksi labeloida melkein kaikkea (esim. kokonaisia kappaleita, osioita, ym.).
Labelin kohteeseen voidaan viitata \ref{avain}-komennolla. \eqref{eq:avain} kaavoille (laittaa sulut viittauksen ympärille)
\pageref{avain}-komennolla saadaan viitattavan kohteen sivunumerokin!


Matematiikka:

Tekstin osana:
Komennolla $Matematiikkaa...$ saadaan matematiikkaa ladottuna tekstin sekaan

Erillisinä kaavoina:
\begin{equation} 
	\label{eq:avain}
	Matematiikkaa...
\end{equation}
Kaava sijoitetaan erikseen ja numeroidaan. Numeroimaton versio saadaan equation*:llä. 
Kaavaan viitataan: \eqref{eq:avain}, esim: ...keskiarvo saadaan kaavalla~\eqref{eq:karvo}~.
Kaavaan saadaan tarvittaessa tekstin tavoin muotoiltuja merkkejä (esim. pilkku ennen missä-sanaa)
komennolla \text{teksti}, esim: \text{,}
Monen rivin kaava saadaan käyttämällä multiline-ympäristöä equation-ympäristön sijaan. Monen rivin kaavassa voidaan
vaihtaa riviä \\-komennolla

\begin{align} 
	3x+2y&=3\\
	-2x&=4y+7\notag %notagilla voidaan jättää yksi yhtälö numeroimatta
\end{align} %align-ympäristöllä saa kohdistettuja yhtälöryhmiä, &-merkillä merkitään kohdistuskohta, \\ vaihtaa riviä numeroi yhtälöt, align*-ympäristö jättää kaikista numeroinnit

f(x) =
\begin{cases} 
0 & \text{ jos } x<0 \\
x & \text{ jos } 0\le x \le 0 \\
1 & \text{ jos } 1 > x
\end{cases} 
%cases-ympäristöllä saadaan paloittain määritelty funktio, sitä pitää käyttää jonkin toisen matikkaympäristön sisällä. Huomaa kiinnitysmerkit & sekä välit \text-komennon sisällä

\begin{pmatrix}
a & b & c \\
d & e & f
\end{pmatrix} 
%pmatrix luo matriisin (pyöreät sulut), bmatrix tekee hakasulkumatriiseja, etc

\left(\begin{array}{c|cc}
a & b & c \\
\hline d & e & f
\end{array}\right) %array on vanha tapa luoda matriiseja, sillä saadaan mm. lohkomatriiseja edellä esitetyllä tavalla

Matematiikan komentoja:
\: -komennolla saadaan matikkatilassa pienen välin merkkien väliin
\, \quad ja \qquad antavat eripituisia välejä matikkatilassa, \! vähentää merkkien välistä tilaa, voidaan käyttää esim moni-integraalien kanssa tiivistämään merkintöjä, esim: \int\!\!\!\int
Kreikkalaiset aakkoset: \ja aakkosen englanninkielinen nimi. Isolla alkukirjaimella iso, pienellä pieni aakkonen
Esim: \alpha pikkualfalle, \Delta isolle deltalle
_ alaindekseille ja ^ yläindekseille, voi yhdistellä vapaasti, eivät mene sekaisin. Pidemmät jutut aaltosulkuihin.
Esim: x^2_{max}
\infty äärettömän merkille
Murtoluvut ja jakolaskut: \frac{}{}, aaltosulkuihin viivan ylä- ja alapuolille tulevat jutut. Osaa venyä fiksusti.
Summamerkki: \sum_{alaindeksi}^{yläindeksi}
Neliöjuuri: \sqrt{}, venyy järkevästi
Järkevästi venyvät sulut: \left(, \right) sekä \left[, \right] ja \left{, \right}  
Keskiarvoa merkitsevä viiva kirjaimen päälle: \bar{}
Plusmiinus: \pm
Noin yhtäsuuri kuin: \approx
Pienempi tai yhtäsuuri kuin(?): \le
Kertomerkki näkyviin: \times tai \cdot
Doo-merkki: \partial
Sopivasti venyvät merkit: \right ja \left (esim. \right), \left[, \right/, etc) pisteellä (.) voi rajata merkkiä

Kuvien liittäminen:

Leijuvainen kuva luodaan figure-ympäristöllä:
\begin{figure}[sijoittelu]
	\begin{center}
		\includegraphics[määreitä?]{kuvatiedoston nimi}
	\end{center}
	\caption{kuvateksti}
	\label{fig:avain}
\end{figure}
Ympäristö luo leijuvaisen kuvan jonka sijoittelusta kääntäjä päättää. Kuva numeroidaan ja 
siihen voi viitata: \ref{fig:avain}. Kuvan sijoitteluun voi (yrittää) vaikuttaa sijoittelumääreillä:
	h tähän
	t jonkun läheisen sivun ylälaitaan
	b jonkun läheisen sivun alalaitaan
	p erilliselle leijuvaissivulle
	! lisäämällä huutomerkin LaTeX ohittaa jotain sijoittelun rajoituksia
Määreitä voi yhdistellä, esim: \begin{figure}[htp]. Järjestys vaikuttaa siihen missä järjestyksessä LaTeX kokeilee eri vaihtoehtoja
\includegraphics-komennolle annettavilla määreillä voi mm. skaalata ja kääntää kuvaa:
"angle="-määre kääntää kuvaajaa annetun astemäärän pos-neg kiertosuunta voimassa, 
"width="-määre skaalaa kuvaajan annetun levyiseksi. Ymmärtää mm. \textwidth-komennon, eli skaalaa kuvan tekstin
levyiseksi. Voi myäs käyttää esim: 0.75\textwidth
HKuvia saa vierekkäin laittamalla peräkkäin \includegraphics-komentoja. Vierekkäin olevissa kuvissa leveyden
täytyy olla riittävän pieni että ne mahtuvat (esim. 0.49\textwidth )!

Taulukot:

Leijuvainen taulukko luodaan tabular-ympäristöllä:
\begin{table}[sijoittelu]
	\caption{otsikko}
	\begin{center}
	\begin{tabular}{taulukon muotoilu}
	Taulukko ja sen sisältö...
	\end{tabular}
	\end{center}
	\label{tab:avain}
\end{table}
Taulukon muotoilu-kohtaan kirjoitetaan taulukon muotoilu sivusuunnassa:
	| pystyviiva
	r oikealle tasattu sarake
	c keskelle tasattu sarake
	l vasemmalle tasattu sarake
Esim: \begin{tabular}{|c|lr|c|}
Taulukko ja sen sisältö-kohtaan kirjoitetaan poikkiviivat: \hline sekä taulukon alkioiden sisällöt &-merkein
eroteltuina. Rivin vaihto rivin viimeisen alkion jälkeen: \\
Esim:	\hline
	Mittaus & Massa (kg) & Tilavuus (\(m^2\)) & Tiheys (\(\frac{kg}{m^2}\)) \\
	\hline
	1 & massa1 & tilavuus1 & tiheys1 \\
	2 & massa2 & tilavuus2 & tiheys2 \\
	.
	.
	.
	\hline
Sarakkeiden tasauksen sai myös tehtyä jonkin merkin (esim. pilkun) kohdalle, mutta sen muoto pitää vielä tarkistaa.
Leijuvaisen taulukon sijoittelussa on käytössä samat määreet kuin kuvien sijoittelussa.
Taulukot numeroidaan ja niihin voi viitata: \ref{tab:avain}
 

Kolme pistettä: \ldots


Omien komentojen luonti:
\newcommand{\RR}{\mathbb{R}} %voidaan luoda vain "lyhennysmerkkejä" TAI
\newcommand{\MB}[2]{\mathbb{#1}-\mathbb{#2}} %näin voidaan luoda "funktion tavoin" toimivia komentoja. Hakasuluissa on argumenttien määrä ja #-merkillä merkataan argumenttinä saadut muuttujat
Käyttö: Ensimmäiselle normaali \RR, jälkimmäiselle esim. \MB{R}{Y} (tässä tapauksessa matikkatilassa)
\renewcommand{}{}-komennolla voi korvata LaTeXin komentoja uusilla.

\tableofcontents %luo sisällysluettelon	
\listoffigures %listaa dokumentissa olevat kuvat
\listoftables %yllättäen listaa taulukot

\selectlanguage{english} %Vaihda kieltä kesken dokumentin.

\url{http://en.wikipedia.org/wiki/Scarlett_Johansson} %lisää linkin nettiosoitteeseen, komento tulee hyperref-paketissa
\href{http://en.wikipedia.org/wiki/Scarlett_Johansson}{Scarlett} %linkki omalla tekstillä
\href{mailto:email_osoite@domain.com}{email_osoite@domain.com} %linkki sähköpostin lähetykseen


\footnote{Sivuhuomautus} %sivun sisäinen viite

Listat:
\begin{itemize} %itemize-ympäristössä listan kohdan eteen tulevaa merkkiä voi muuttaa item-kohtaisesti hakasulkeisiin tulevalla merkillä (ks. seuraavassa)
	\item[-] Esko
	\item[a] on 
	\item kiki 
\end{itemize}

\begin{enumerate}[custommuotoilu] %Tunnistaa numerointimerkit 1, I, i, A ja a. Merkkien escapaus esim. [L{ii}te A:]
	\item Numeroitava lista
\end{enumerate}

pdfpages-paketilla voit lisätä tiedostoon kokonaisia sivuja toisesta pdf-tiedostosta. Sivut lisätään \includepdf-komennolla . Optioina voidaan antaa lisättävät sivut pages=1,2,5 tai pages=- (kaikki sivut)
\includepdf[pages=-]{Mittauspoytakirja.pdf} 
lisäksi mm. optiolla voidaan suorittaa komentoja joka sivulla: pagecommand={\komentoja}

%Easylist-paketti auttaa lukuisia tasoja sisältävien listojen laatimisessa (tason sisennysmerkki annetaan paketin 
%argumenttina (nyt ampersand) 
%itse listan tyyliä voi vaihtaa ympäristön argumentissa: itemize on perus itemizen tyyli, enumerate perus enumeraten (1. -> (a) -> i. -> A.)
\begin{easylist}[itemize]
	& Eka taso
	&& Toka taso
	&&& Kolmas taso
	&&&& Neljäs taso
\end{easylist}
	



\begin{verbatim}
 Muotoilematonta tekstiä:
        esko     on     kiki.
\end{verbatim}
Rivillä verbatim ympäristö: \verb|C:\Windows\|
\verb-komennossa ensimmäinen merkki on rajoitusmerkki (esim. \verb+ ... + ), \verb-komentoa tarvitsee yleensä
kun pitää latoa kenoviivoja.

Kahvitahrojen lisääminen (coffee3):
Neljä erilaista kahvitahraa:
\cofeAm{läpinäkyvyys}{koko}{kierto}{xoff}{yoff} %270 asteen kupin jälki
\cofeBm{läpinäkyvyys}{koko}{kierto}{xoff}{yoff} %60 asteen kupin jälki
\cofeCm{läpinäkyvyys}{koko}{kierto}{xoff}{yoff} %kaksi vaaleaa roisketta
\cofeDm{läpinäkyvyys}{koko}{kierto}{xoff}{yoff} %värillinen tuplatahra

Komentojen argumentteina on tahran läpinäkyvyys väliltä [0,1], tahran kokoskaala (normaali on 1),
tahran kiertoaste väliltä [0,360] sekä paikka offsettina keskeltä (xoff, yoff).
HUOM! Tahran lisäävä komento laitetaan sen sivun koodin sekaan, jonne se halutaan. 
HUOM! Tahrat tulevat suurinpiirtein keskelle sivua ja ovat tekstin alla VAIN jos komento on 
kirjoitettu sen osion (esim. \section) sisälle, jonka sivulla ne ovat. Tahrakomennon paikka voi siis
joutua muokkaamaan / pakottaamaan sivunvaihdon osion kohdalle.


\end{comment}

\end{document}
